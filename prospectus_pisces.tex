\documentclass[12pt, preprint,letterpaper]{article}
\usepackage{epsfig}
\usepackage{xspace}
\usepackage{amsmath}
\usepackage{mathrsfs}
\usepackage{xcolor}
\usepackage{graphicx}
\usepackage{natbib}
\usepackage{hyperref}
\usepackage[utf8]{inputenc}
\graphicspath{ {figures/} }
%%for list of tables
\usepackage{etoolbox}
\makeatletter
\patchcmd{\@caption}{\csname the#1\endcsname}{\csname fnum@#1\endcsname:}{}{}
\renewcommand*\l@figure{\@dottedtocline{1}{1.5em}{4.5em}} % default for 3rd arg: 2.3em
\let\l@table\l@figure % as in article.cls
\makeatother
%% end list of tables
%%
% Begin of document
\begin{document}
%
%
%#******************************************************************************
%#==============================================================================
%#          Beginning: Preamble page
%#==============================================================================
%#******************************************************************************
%
\thispagestyle{empty}
\begin{center}
\textbf{Effects of Wavelength Dependence on Galaxy Shear Measurement} \\

\vspace{2cm}
PH.D. THESIS PROSPECTUS BY BHISHAN POUDEL \\
(ADVISOR: ASSOCIATE PROF. DOUGLAS CLOWE) \\

\vspace{3cm}
\it{
Department of Physics and Astronomy\\
Ohio University\\
Athens, OH 45701\\
\vspace{2cm}
bp959314@ohio.edu\\
}


\vspace{3cm}
\it{
Prof. Douglas Clowe\,  \hspace{2mm} (Department of Physics \& Astronomy)\\
Prof. Ryan Chornock\,  \hspace{2mm} (Department of Physics \& Astronomy)\\
Prof. Hee Jong Seo\, \hspace{3mm} (Department of Physics \& Astronomy)
}

\end{center}
%
%
%#******************************************************************************
%#==============================================================================
%#          Table of contents
%#==============================================================================
%#******************************************************************************
%
%
\clearpage
\tableofcontents
\listoffigures
\listoftables
\clearpage{}
%
%
%#******************************************************************************
%#==============================================================================
%#          List of Acronyms
%#==============================================================================
%#******************************************************************************
%
\thispagestyle{empty}
\begin{center}
\textbf{LIST OF ACRONYMS}
\end{center}
\textbf{CDM}    \hspace{9.5mm} Cold Dark Matter \\
\textbf{CMB}    \hspace{10mm} Cosmic Microwave Background \\
\textbf{DE}     \hspace{15mm} Dark Energy \\
\textbf{DM}     \hspace{14mm} Dark Matter \\
\textbf{FWHM}   \hspace{6mm} Full Width Half Maximum \\
\textbf{GR}     \hspace{15mm} General Relativity \\
\textbf{HST}    \hspace{13mm} Hubble Space Telescope \\
\textbf{IMCAT}  \hspace{7mm} Image and Catalogue Manipulation Software \\
\textbf{LSS}    \hspace{15mm} Large Scale Structure \\
\textbf{LSST}   \hspace{12mm} Large Synoptic Survey Telescope \\
\textbf{PSF}    \hspace{14mm} Point Spread Function \\
\textbf{SED}    \hspace{14mm} Spectral Energy Distribution \\
\textbf{WFIRST} \hspace{4mm}  Wide-Field Infra-red Survey Telescope \\
%longest word wfirst is 6 letter long and make it 4mm wide.
%one letter less is about 3mm longer space.

\newpage
%
%
%#******************************************************************************
%#==============================================================================
%#          Section 1: Introduction
%#==============================================================================
%#******************************************************************************
%
%
\section{Introduction}\label{sec:sec1}
According to standard model of cosmology dark matter accounts for  28.5\% of total mass energy content of the Universe.
It's mainly dominated in Galaxy and Galaxy Clusters.
Anisotropy in the Cosmic microwave background,cosmic
structure formation, galaxy formation
and evolution suggest the presence of dark matter.
As it does not interact with electromagnetic radiation and
visible matter the method it can  be detected is by observing
it's effect on ordinary baryonic matter.
%
%#******************************************************************************
%#==============================================================================
%#          Section 1: Introduction
%#==============================================================================
%#******************************************************************************
%
%
Gravitational Lensing provides us a way
to see how dark matter along with visible matter is distributed
in a galaxy or galaxy cluster.
This is supported by Einstein's general theory of relativity
which predicts the deflection of light in a gravitational field
produced by a massive object. (refer to \cite{fabian12})
\ref{sec:sec1}
%
%
%#------------------------------------------------------------------------------
%#          Subsection 1a: Formulation of Gravitational Lensing
%#------------------------------------------------------------------------------
%
%
\subsection{ Formulation of Gravitational Lensing}\label{subsec:lensing}
I have to write here.

\subsubsection{Einstein's Deflection Angle}\label{subsubsec:lensing}
% aug 10, barSch  page 326
General Relativity predicts that when the beam of light passes throug near
the massive objects the rays are bent. If the massive object of mass $M$ is at a
perpendicular distance $\xi$ (i.e. impact parameter) then the deflection for point
mass is given by
\begin{equation}\label{[eq:delection_eqn]}
\hat{\alpha} = \frac{4GM}{c^2\xi}
\end{equation}
This equation (\ref*{[eq:delection_eqn]} ) is valid only when the angle
 $\hat{\alpha} <<1  $ . In  our field of interest gravitational lensing it is always the
 case and the deflection angle is always very very small. For quantitative purpose,
 we can calcuate the value for solar limb light deflection.
 We set following values in the above equation (\ref*{[eq:delection_eqn]} ):\
 $M = M_{\odot}$, $ R = R_{\odot}$ and calculation results in \
 \begin{equation}\label{[eq:delection_eqn]}
 \hat{\alpha} = 1.74''
 \end{equation}

 \begin{figure}[h!]\label{[fig:lensing]}
    \centering
    \includegraphics[width=0.5\textwidth]{figures/lensing2.jpeg}
    \caption{Simple sketch to gravitational lensing. (Bartlemann and Schneider 2001)} % cite this
\end{figure}

We want to probe the mass of the lens which has the angular diameter distance
 $D_d$ with redshift $z_d$ and lies between the observer and the source background mass.
 The lens mass can be visible or dark matter however, the
 source mass is visible objects such as galaxies, galaxies-cluseter, QUASAR, and so on.
 The source has angular diameter distance $D_s$ with a redshift $z_s$.
 In absence of lens plane source will make an angle $\theta$ to the observer,
 and in presence of lens the source will make an angle $\beta$ to the observer.
 The deflection angle is $\alpha$.








%
%
%#------------------------------------------------------------------------------
%#          Subsection 1a: Creation of PSF
%#------------------------------------------------------------------------------
%
%
\subsection{Creation of PSF}\label{subsec:}
I have to write this chapter.







%
%
%#******************************************************************************
%#==============================================================================
%#          Section 2:  Future Projects
%#==============================================================================
%#******************************************************************************
%
%
\section{Future Projects}\label{sec:sec2}
 \begin{figure}[h!]
 \centering
  \includegraphics[width=0.5\textwidth]{convolve.png}
 \caption{This is caption.}
 \end{figure}
\newpage{}
%
%
%#******************************************************************************
%#==============================================================================
%#          Section 3: Timeline
%#==============================================================================
%#******************************************************************************
%
%
\section{Timeline}
I intend to complete all elements of this project within two years.
An outline of the projected time line is presented in the table below.
The dates and activities displayed are subject to change should
unforeseen and interesting developments occur during the projects that
may require special attention and time.
\begin{table}[h]
\caption{Time-line}
\vspace{5mm}
\centering
        \begin{tabular}{|c|c|c|}
            \hline
            Start&Finish&Activity\\
            \hline
            Sep. 2016& March 2017 &  \\
            \hline
            April 2017 & Sep. 2017 &   \\
            \hline
            Sep. 2017 & March 2018 & Preparation of manuscripts for publication \\
            \hline
            March 2018 & Aug. 2018& Writing and defending the dissertation   \\
            \hline
            \end{tabular}
\end{table}
%
%
%#******************************************************************************
%#==============================================================================
%#               Bibliography and End Document
%#==============================================================================
%#******************************************************************************
%
%
\newpage
\bibliographystyle{plain}
\bibliography{bib/prospectus}
\end{document}
